\documentclass{article}
\usepackage[utf8]{inputenc}
\usepackage[version=4]{mhchem}

\title{CHEM IA DRAFT}
\author{Peter Wang}
\date{November 2021}

\begin{document}

\maketitle
\section{Introduction}
Ever since I immigrated to Canada as a child, it has become my family’s tradition to visit a new ski resort every winter. However, I observed over the years that when I brought my phone along to take pictures of the landscape, it would immediately shutdown as soon as I took it out of my coat pocket. When I tried to restart the device, it would indicate that the device had no battery even though I had fully charged it beforehand. It was not until my final year of high school when I learned about the spontaneity of reactions and how it was indicated by a measurement known as the Gibbs Free Energy. In the energetics unit, I also learned that the Gibbs Free Energy of a reaction is dependent on the temperature at which the reaction occurs, and this led me to wonder how temperature affected the Gibbs Free Energy of a reaction.

\section{Investigation}
\subsection{Reaction Under Study}
The reaction studied in this investigation is the dissolution of Ammonium Nitrate within water, which produces Ammonium cations and Nitrate anions and absorbs heat in the form of an endothermic reaction.
$$ \ce{NH4NO3 (s) + H2O (l) -> NH4+ (aq) + NO3- (aq)} $$

\subsection{Research Question}
\textit{The effect of absolute temperature on the Gibbs Free Energy of the dissolution of Ammonium Nitrate in Water at 298K, 303K, 308K, 313K and 318K.}

\subsection{Background Information}
The Gibbs Free Energy of a reaction is the amount of Free Energy available to do work and combines the enthalpy and entropy values into a single state function. The Gibbs Free Energy is given as the difference between the enthalpy and the product of entropy and absolute temperature. As indicated by the equation, as the absolute temperature changes, so will the Gibbs Free Energy of the reaction.
$$
\Delta G = \Delta H - T \Delta S
$$

However, both the enthalpy as well as the entropy of the reaction is also dependent on the temperature at which the reaction occurs. The specific enthalpy of a system is given as the sum between the internal energy and the product of the pressure and volume. Therefore, as the temperature increases, so will the overall internal energy of the system and the enthalpy of the reaction.
$$
H = E + PV
$$



\section{Experimental Method}
\subsection{Materials}
\begin{itemize}
    \item 100.0 g Ammonium Nitrate 
    \item Distilled Water
    \item Insulated Coffee Cup (with Lid)
    \item Pasco-Thermometer
    \item Laptop with Bluetooth Connection
    \item Scrap Paper
    \item Stirring Rod
\end{itemize}

\subsection{Experimental Procedure}
The experiment is a coffee cup calorimetry experiment where the temperature over time of the reaction is measured using a digital Pasco-Thermometer. The temperature can be graphed over time and the corrected final temperature can be corrected by extrapolating the temperature gain from the environment. 
\begin{enumerate}
    \item Measure out 4.0 g of Ammonium Nitrate in a Paper weighing boat on the two digit electronic balance. 
    \item Measure 100 mL of water using a 50 mL Graduated Cylinder
    \item Heat up the water to 25°C using a hot plate 
    \item Add the heated water into the coffee cup calorimeter and add the Ammonium Nitrate
    \item Add the Pasco-Thermometer and start recording the temperature in the SparkVue software.
    \item Stir the reaction as it progresses to ensure that the Ammonium Nitrate is being dissolved in the water
    \item Once the temperature stops decreasing in the SparkVue software, let the Pasco-Thermometer to continue recording for another 10 seconds.
    \item Repeat steps 1-7 for three trials at 30°C, 35°C, 40°C, 45°C and 50°C.
\end{enumerate}

\subsection{Experimental Setup}
The experimental setup can be demonstrated by the picture below:

\end{document}
